\chapter{Experiments}\label{ch:experiments}
Each component of the system described in the previous chapter has its own influence in the performance of the whole method proposed in this work. We have yet to discover if their inclusion in the 3D human pose estimation pipeline is justified. Furthermore, we have to show how our lightweight algorithm compares with more complex solutions in the literature, and discuss if we have reached a fair trade-off between computational costs and accuracy.

For that purpose, we have performed both quantitative and qualitative evaluations. In this chapter, we show and discuss the results obtained for both of them. We will start presenting the publicly available dataset we have used as benchmark for both 2D and 3D pose estimations. Then, we will define the figures of merit used for evaluating 2D estimations and analyze the different results achieved by each of the methods described in Section~\ref{sec:2d_estimation}. Regarding 3D estimation, relevant figures of merit will be defined too and used to evaluate the effects of the mechanisms described in Section~\ref{sec:3d_estimation}. Quantitative evaluation will be concluded with a thorough comparison of our method against a \gls{sota} solution, both in terms of accuracy and computational costs. Last but not least, a demonstration of our system working in real-time will be presented, which will help us to validate whether our system is suitable as a component for robotics applications.

\section{Benchmark dataset}\label{sec:benchmark_dataset}
For quantiative evaluation, we will use as benchmark the publicly available \gls{bmhad}~\cite{ofli2013berkeley}. This database is comprised of synchronized data from multiple color cameras, \emph{Kinect} sensors, accelerometers, microphones and a \gls{mocap} system. From all of these, we will use \emph{Kinect} RGBD videos as input data for evaluation and \gls{mocap} data as ground-truth. The database is divided into scenes recorded from 12 subjects performing 11 different actions, with 5 repetitions per action. In other words, a total number of 660 scenes are provided. Each of these scenes has a varying length, adding up to 82 minutes of recording. As the database provides data from two \emph{Kinect} cameras capturing the scene from different viewpoints, we will include a total number of 1320 scenes in our analysis. In Figure~\ref{fig:bmhad_setup}, a diagram of the set-up used for capturing these scenes is presented.

\begin{figure}[h]
    \centering
    \includegraphics[width=\textwidth]{figures/bmhad_setup.png}
    \caption{Diagram of the data acquisition set-up used for building BMHAD~\cite{ofli2013berkeley}.}
    \label{fig:bmhad_setup}
\end{figure}

After studying other public datasets published in the last years, we found \gls{bmhad} to be the only one that satisfies our needs. First of all, most of the \emph{large-scale} datasets in the literature are focused on 2D pose estimation (\eg MPII~\cite{andriluka20142d} and \emph{Leeds Sports Pose}~\cite{Johnson10}). Among those containing 3D annotations, it is very hard to find registered RGB and depth images, which is a major requirement for our algorithm. That is the case of the prominent \emph{Human3.6M}~\cite{ionescu2013human3} and \emph{HumanEva}~\cite{sigal2010humaneva} datasets. Besides that, a lot of datasets with 3D information are more oriented to serve as input for action recognition tasks, and the joint locations are estimated with \glspl{sdk} provided by the manufacturers of the RGBD cameras, instead of high accuracy locations as provided by a professional \gls{mocap} system. Such is the case of the \emph{Cornell Activity Datasets}~\cite{sung2011human}. Taking all of these factors into account, the only \emph{large-scale} dataset that provides high-quality annotated joint locations registered and synchronized with both RGB and depth images is the \gls{bmhad}.

Originally, the \gls{bmhad} dataset provides labels for 21 joints extracted from the raw \gls{mocap} data. However, our final estimations depend on the 2D keypoints given by the models presented in Section~\ref{sec:2d_estimation}, which have been trained with samples from the MPII dataset. In order to make our results comparable with \gls{bmhad} labels, we have reduced the number of joints provided to 12: head, shoulders, elbows, hands, pelvis, knees and feet. For head and pelvis, some modifications to the original labels have been performed in order to match \gls{bmhad} and MPII definitions. In the case of the head label, the \textit{HeadEnd} and \textit{Head} labels in \gls{bmhad}, and the \textit{head top} and \textit{upper neck} labels in MPII  have been averaged, respectively, in order to get a unique comparable head keypoint. In the case of the pelvis label, the \textit{LeftUpLeg} and \textit{RightUpLeg} labels in \gls{bmhad} and the \textit{l hip} and \textit{r hip} labels in MPII have been averaged as well. For the latter, merging left and right hips labels has been necessary because of slightly different definitions in terms of what is considered \textit{hip} or \textit{UpLeg} in each of the datasets, looking much more similar between them after the averaging process. In Figure~\ref{fig:bmhad_labels}, a visual comparison between the original \gls{bmhad} labels and the result of our transformation is shown.

\begin{figure}[h]\centering
    \begin{subfigure}{0.47\textwidth}\centering
        \includegraphics[height=5.75cm]{figures/labels_original.png} 
        \caption{Original BMHAD labels (red crosses).}
        \label{subfig:cp_general}
    \end{subfigure}
    \begin{subfigure}{0.51\textwidth}\centering
        \includegraphics[height=5.75cm]{figures/labels_modified.png}
        \caption{Modified labels with joints linked.}
        \label{subfig:cp_particular}
    \end{subfigure}
    \caption{In order to compare our results with the labels provided by BMHAD, we have transformed the original keypoints shown in (a) to the simpler configuration shown in (b). Please note that in (a) markers used for MoCap and final keypoints are shown in blue and red, respectively. Also, (a) and (b) show the same moment in time, but captured with different cameras.}
    \label{fig:bmhad_labels}
\end{figure}

It is worth mentioning that a major limitation of using \gls{bmhad} for assessing our performance is its lack of scenes \emph{in the wild}. This is an issue with no easy solution for the field of 3D human pose estimation, as precise labels can only be captured by means of cumbersome \gls{mocap} systems, which can hardly be used in more natural scenarios. Besides that, the dataset only provides two different points of view, so further research in terms of how our system performs under more intricate perspectives can only be done qualitatively. Another issue we have discovered is that, for frames with fast motions, \gls{mocap} labels might be displaced, as it can be seen in the left hand of the subject in Figure~\ref{fig:wrong_hands}. Even though this effect might slightly drop our accuracy, it will equally affect every method tested and so comparisons will still be well-founded.

\begin{figure}[h]\centering
    \begin{subfigure}{0.49\textwidth}\centering
        \includegraphics[height=5.75cm]{figures/2d_gt.png} 
        \caption{Original BMHAD label projected in 2D.}
        \label{subfig:2d_gt}
    \end{subfigure}
    \begin{subfigure}{0.49\textwidth}\centering
        \includegraphics[height=5.75cm]{figures/2d_cpm.png}
        \caption{Joints estimated using CPMs.}
        \label{subfig:2d_cpm}
    \end{subfigure}
    \caption{In scenes with rapid movements, such as \textit{a01}, BMHAD labels might suffer some displacements. In (a), this effect is demonstrated to be significant for the left hand label. In (b), the corresponding estimation is shown.}
    \label{fig:wrong_hands}
\end{figure}

In order to provide a more intuitive interpretation of the results, the different actions considered in the dataset have been classified in terms of their dynamics and the occlusion degree of the human body joints, as follows:

\begin{itemize}
    \item \textbf{High dynamics and heavily occluded}: Bending (\textit{a03}) and throwing ball (\textit{a08}). From now on, results for these actions will be presented in \textit{red}.
    \item \textbf{High dynamics and slightly occluded}: Jumping in place (\textit{a01}), jumping jacks (\textit{a02}) and boxing (\textit{a04}). From now on, results for these actions will be presented in \textit{orange}.
    \item \textbf{Low dynamics and heavily occluded}: Sit down then stand up (\textit{a09}), sit down (\textit{a10}) and stand up (\textit{a11}). From now on, results for these actions will be presented in \textit{green}.
    \item \textbf{Low dynamics and slightly occluded}: Waving with two hands (\textit{a05}), waving with one hand (\textit{a06}) and clapping hands (\textit{a07}). From now on, results for these actions will be presented in \textit{blue}.
\end{itemize}

\section{2D Estimation quantitative evaluation}\label{sec:2d_estimation_evaluation}
Following the approach of many works in the literature, the person detection and subsequent pose estimation tasks are decoupled. For evaluation purposes, we will assume that a bounding box fitted around the subject can be inferred from the ground-truth. More precisely, we compute the center of the human pose as the average location between all the joints and crop a squared region around it, which is 1.25 times bigger than the biggest difference between joints in \(x\) or \(y\) dimensions. It is important to note that these bounding boxes have been resized for each method evaluated according to the sizes defined in their respective articles: 368 pixels for \glspl{cpm} and 256 pixels for both \gls{sh} and \glspl{cp}.

As mentioned in Section~\ref{sec:benchmark_dataset}, we will use \gls{bmhad} for evaluation. However, as \gls{bmhad} only provides labels in 3D coordinates, we have projected the 3D joint locations into the RGB video sequences using the camera calibration parameters included in the dataset. For this evaluation, both our estimations and the ground-truth labels have been converted to the 12 joints model described in the previous section.

\subsection{Figures of merit}\label{subsec:2d_metrics}
The performance of the 2D pose estimation approaches presented in Section~\ref{sec:2d_estimation} has been evaluated according to the \gls{pckh} indicator~\cite{andriluka20142d}. According to the \gls{pckh} score, the estimation of the joint position is considered to be correct if the distance between the estimated and the ground-truth joint locations is below a threshold dependant on the head segment length. The threshold is usually indicated after the \textit{at} symbol, thus for a \gls{pckh}@0.1, we would consider as correct keypoints those with an error lower than the 10\% of the head link size.

For the results presented in this report, the head segment length is obtained for each frame from the corresponding ground-truth. The \gls{pckh} score is computed for each joint considering all the frames in the same video sequence for thresholds ranging between 0 and 1 head segment lengths. Regarding missing joints, they are considered to be wrong regardless of the threshold. 

\subsection{Experimental results}\label{subsec:2d_experimental_results}
For the 2D estimation evaluation, results per joint and action are presented for the three methods tested: \glspl{cpm}~\cite{Wei2016-rb}, \gls{sh}~\cite{Newell2016-cy} and \gls{cp}~\cite{Gkioxari2016-ix}. As it can be seen in Figure~\ref{fig:pckh_2d}, these methods show similar performance with respect to each other regardless of the human joint. Having said that, \glspl{cpm} perform slightly better than the other two for every joint if we set the threshold to the length of one head segment (\gls{pckh}@1), with a mean score of 95.85\% of parts correctly detected (see Table~\ref{tab:pckh_2d}). Regarding the differences in performance between joints, hand locations are harder to disambiguate than any other joint, with a mean \gls{pckh}@1 below 90\% in every method tested. This can be justified by the higher level of occlusion they might have in comparison with other joints. Besides that, hands are the joints with higher dynamics in the actions included in the dataset, and labels might be displaced as we previously mentioned in Section~\ref{sec:benchmark_dataset}. Figure~\ref{fig:wrong_hands} presents an example of a misplaced hand label, along with our estimation.

\begin{figure}[h]
    \centering
    \includegraphics[width=\textwidth]{figures/pckh_2d.png}
    \caption{Quantitative 2D pose estimation results on BMHAD per joint, using as figure of merit PCKh 2D~@~0-1~(\%). }
    \label{fig:pckh_2d}
\end{figure}

\begin{table}[!ht]  
  \centering
  \resizebox{\textwidth}{!}{\begin{tabular}{l||ccccccc|c}
    \hline
    & Head & Shoulders & Elbows & Hands & Pelvis & Knees & Feet & TOTAL \\
    \hline
    \glspl{cpm} & \textbf{98.73} & \textbf{97.34} & \textbf{93.28} & \textbf{88.20} & \textbf{99.31} & \textbf{98.03} & \textbf{99.26} & \textbf{95.85} \\
    \gls{sh} & 96.06 & 95.11 & 91.89 & 86.71 & 96.26 & 96.04 & 96.74 & 93.77 \\
    \glspl{cp} & 96.11 & 95.12 & 88.93 & 80.97 & 95.88 & 95.35 & 94.80 & 91.86 \\
    \hline
  \end{tabular}}
  \caption{Quantitative 2D pose estimation results on BMHAD per joint, using as figure of merit PCKh~2D~@~1~(\%). Values in bold correspond to the best results achieved for each category.}
  \label{tab:pckh_2d}
\end{table}

\begin{table}[!ht]  
  \centering
  \resizebox{\textwidth}{!}{\begin{tabular}{l||oorobbbrggg|c}
    \hline
    & a01 & a02 & a03 & a04 & a05 & a06 & a07 & a08 & a09 & a10 & a11 & TOTAL \\
    \hline
    \glspl{cpm} & 97.98 & 96.90 & \textbf{87.49} & \textbf{97.04} & \textbf{98.23} & \textbf{97.26} & \textbf{99.30} & \textbf{96.25} & \textbf{98.04} & \textbf{97.14} & \textbf{98.16} & \textbf{95.85} \\
    \gls{sh} & \textbf{98.35} & \textbf{98.40} & 80.42 & 96.30 & 97.42 & 95.80 & 97.91 & 93.96 & 96.77 & 94.73 & 96.13 & 93.77 \\
    \glspl{cp} & 96.55 & 94.07 & 77.69 & 95.06 & 94.05 & 93.71 & 97.40 & 93.34 & 96.19 & 92.60 & 95.64 & 91.86 \\
    \hline
  \end{tabular}}
  \caption{Quantitative 2D pose estimation results on BMHAD per action, using as figure of merit PCKh~2D~@~1~(\%). Values in bold correspond to the best results achieved for each category. Columns are colored following the convention established in Section~\ref{sec:benchmark_dataset}.}
  \label{tab:pckh_2d_action}
\end{table}

\begin{table}[!ht]  
  \centering
  \resizebox{\textwidth}{!}{\begin{tabular}{l||ro|gb}
    \hline
    & \multicolumn{2}{c|}{High Dynamics} & \multicolumn{2}{c}{Low Dynamics} \\
    \cline{2-5}
    & Heavily Occluded & Slightly Occluded & Heavily Occluded & Slightly Occluded \\
    \hline
    \glspl{cpm} & \textbf{91.87} & 97.31 & \textbf{97.78} & \textbf{98.26} \\
    \gls{sh} & 87.19 & \textbf{97.68} & 95.88 & 97.04 \\
    \glspl{cp} & 85.52 & 95.22 & 94.81 & 95.05 \\
    \hline
  \end{tabular}}
  \caption{Quantitative 2D pose estimation results on BMHAD per action type,  using as figure of merit PCKh~2D~@~1~(\%). Values in bold correspond to the best results achieved for each category. Columns are colored following the convention established in Section~\ref{sec:benchmark_dataset}.}
  \label{tab:pckh_2d_action_type}
\end{table}

The relation between the actions performed in the evaluated scenes and the results in the 2D pose estimation for each method are presented in both Table~\ref{tab:pckh_2d_action} and Table~\ref{tab:pckh_2d_action_type}. In the latter, the actions have been grouped in terms of dynamics and occlusion, as described in Section~\ref{sec:benchmark_dataset}. Taking a look at Table~\ref{tab:pckh_2d_action}, it can be concluded that, in general, \glspl{cpm} performs better than SH and CPs, as it gets a significantly better score for most of the actions. Besides, as presented in Table~\ref{tab:pckh_2d_action_type}, all methods perform similarly for actions with low dynamics and/or slightly occluded joints, with a difference of only $\pm3\%$ in the final score. However, there is a much greater gap in performance when the scenes with the hardest actions are evaluated, \ie heavily occluded actions with high dynamics. For this kind of actions, \glspl{cpm} get a \gls{pckh}@1 of 91.87\%, while SH and CPs achieve a 87.19\% and 85.52\% of correct parts, respectively. This is a very relevant result . As \gls{bmhad} has been captured in a controlled environment, determining which method performs better under adverse circumstances can give us an idea of how these methods would perform in the real world. Figure~\ref{fig:2d_comparison} displays an example where \glspl{cpm} outperforms its competitors when estimating a very challenging pose.

\begin{figure}[h]\centering
    \begin{subfigure}{0.32\textwidth}\centering
        \includegraphics[height=7.cm]{figures/chained_failed.png} 
        \caption{CPs failed estimation.}
        \label{subfig:chained_failed}
    \end{subfigure}
    \begin{subfigure}{0.32\textwidth}\centering
        \includegraphics[height=7.cm]{figures/stacked_failed.png}
        \caption{SH failed estimation.}
        \label{subfig:stacked_failed}
    \end{subfigure}
    \begin{subfigure}{0.32\textwidth}\centering
        \includegraphics[height=7.cm]{figures/cpm_right.png}
        \caption{CPMs accurate estimation.}
        \label{subfig:cpm_right}
    \end{subfigure}
    \caption{In challenging scenes like \textit{a03}, with high dynamics and occlusions, CPMs (c) outperform CPs (a) and SH (b) methods. }
    \label{fig:2d_comparison}
\end{figure}

\section{3D Estimation quantitative evaluation}\label{sec:3d_estimation_evaluation}
For 3D estimation, we will rely again on the labels provided by the \gls{mocap} data included in the \gls{bmhad} dataset. For simplicity, all of the estimations generated in this work are given in camera coordinates. Taking that into account, the corresponding \gls{bmhad} labels have been transformed from world to camera coordinates for the evaluation. Also, following the same strategy described in the previous section, both estimations and ground-truth labels have been further transformed to match our 12 joints model defined in Section~\ref{sec:benchmark_dataset}. It is important to note that these transformations have been applied on the final 3D estimations, but the 2D estimations used as input have preserved their original MPII joints definition. In that way, we can fairly compare our results with those provided by other methods that might have been trained for 3D estimation with different label definitions.

\subsection{Figures of merit}
The performance of the 3D estimation methods presented in this work has been evaluated using the \gls{mpjpe} and the 3D version of the \gls{pckh} score, already described in Section~\ref{subsec:2d_metrics}. \gls{mpjpe} is simply defined as the average \emph{Euclidean} distance between the prediction for the joints location  in the skeleton and their respective ground-truths. Regarding the \gls{pckh} score, we use the same definition presented in the previous section, being the only difference that the head link segment length will be extracted from the three-dimensional data in terms of real-world distance, instead of pixels in a 2D image. 

\subsection{Experimental results}
In order to assess the effectiveness of our method for 3D estimation described in Section~\ref{sec:3d_estimation}, we compare the \gls{mpjpe} obtained using as baseline the direct 3D estimation from the 2D keypoints, without any filtering. For this test, we have used \glspl{cpm} as 2D estimator, since it is the most accurate among the evaluated ones, as we demonstrated in Section~\ref{subsec:2d_experimental_results}. We have selected a subset of 132 scenes from the \gls{bmhad} dataset, containing every action performed by every subject, but only its first repetition and captured with the first of the \emph{Kinect} sensors.

As it can be shown in Table~\ref{tab:mpjpe_joint_baseline}, the overall performance of the 3D pose estimation improves significantly with the inclusion of the processing steps described in Section~\ref{sec:3d_estimation}. Specifically, the total \gls{mpjpe} gets reduced in half. It is important to note that this difference is mostly driven by the results obtained for the joints that belong to the upper body. For most of the actions in the dataset, these joints are the ones with higher dynamics and self-occlusions, which brings to light how our algorithm is able to perform under these circumstances when compared with a direct inference. When \textit{pelvis}, \textit{knees} and \textit{feet} get compared, the baseline results are slightly better. As these joints are less challenging to estimate throughout the scenes in the dataset, our mechanisms can actually distort their locations to some extent. For instance, if both the depth map and the 2D estimation are completely accurate, our minima filter could pick a location which is closer to the camera that it actually is. Nonetheless, this mechanism is very effective for correcting errors when the conditions are not ideal.

\begin{table}[!ht]  
  \centering
  \resizebox{\textwidth}{!}{\begin{tabular}{l||ccccccc|c}
    \hline
    & Head & Shoulders & Elbows & Hands & Pelvis & Knees & Feet & TOTAL \\
    \hline
    Baseline & 1413.2 & 194.5 & 164.0 & 355.5 & \textbf{174.9} & \textbf{90.5} & \textbf{65.8} & 277.4 \\
    Ours & \textbf{104.2} & \textbf{146.3} & \textbf{125.6} & \textbf{137.6} & 198.2 & 103.7 & 79.0 & \textbf{123.9} \\
    \hline
  \end{tabular}}
  \caption{Quantitative 3D pose estimation results on BMHAD per joint, compared against our baseline, using as figure of merit MPJPE (mm). Values in bold correspond to the best results achieved for each category.}
  \label{tab:mpjpe_joint_baseline}
\end{table}

In that sense, it is worth discussing in detail the great difference for the \textit{head} joint. In that particular case, as we explained in Section~\ref{sec:benchmark_dataset}, the labels get transformed after 3D estimation, but the original 2D estimations are yielded in MPII format, which gives two different labels for the head of the subject: \textit{upper neck} and \textit{head top}. The latter is placed right above the head, which makes the 3D estimation consistently fail if no further processing is applied. Thanks to our minima filter, this error gets dramatically reduced. An example of such case can be seen in Figure~\ref{fig:head}.

\begin{figure}[h]
    \centering
    \includegraphics[width=\textwidth]{figures/head.png}
    \caption{Small deviations in the 2D estimation can lead to huge differences in depth. At the left side of the figure, the depth image with the region of interest around the \textit{head top} joint. At the right side, we can see highlighted in red the original 2D estimation and in green the point we used for depth estimation after minima filter correction.}
    \label{fig:head}
\end{figure}

In order to better understand the nature of the error we get in the estimation, we have analyzed how it affects each dimension (\(X\), \(Y\), \(Z\)) independently. For this purpose, we have plotted the module of the error per joint and dimension for our 3D estimation, using once again \glspl{cpm} as 2D estimator. As we can see in Figure~\ref{fig:mpjpe_by_dim}, most of the error in the evaluation comes from the \(Z\) dimension, \ie from the estimated depth or distance to the camera. As demonstrated in Section~\ref{subsec:2d_experimental_results}, the 2D estimators tested perform very accurately, with \gls{pckh} scores up to 95\%. However, very slight deviations in the 2D estimation and self-occlusions can make huge differences in the estimated depth.

\begin{figure}[h]
    \centering
    \includegraphics[width=\textwidth]{figures/mpjpe-xyz-ours.png}
    \caption{Quantitative 3D pose estimation results on BMHAD per joint, using as figure of merit the module of the difference between estimations and ground-truth for each dimension (mm).}
    \label{fig:mpjpe_by_dim}
\end{figure}

\subsection{Comparison with other methods}
Our approach follows the strategy proposed by Zimmermann \etal\cite{Zimmermann2018-sn}, in the sense that our estimations also rely on registered RGB and depth images. For detecting the 2D keypoints in the RGB images, they use a model provided by the \emph{OpenPose} library~\cite{cao2018openpose} with fixed weights. Then, an occupancy voxel grid with a resolution of about 3 cm is computed by transforming the depth map into a point-cloud and, using the calibration matrix, the initial 3D coordinates of the keypoints are estimated. The voxel grid and the 2D likelihood maps are processed by a 3D CNN which optimizes the \emph{Sum of Squared Errors} between the ground-truth and the estimated location in 3D. An overview of the architecture proposed by Zimmermann \etal can be seen in Figure~\ref{fig:zimmermann}.

\begin{figure}[h]
    \centering
    \includegraphics[width=\textwidth]{figures/zimmermann.png}
    \caption{Overview of the solution introduced by Zimmmermann \etal for 3D human pose estimation~\cite{Zimmermann2018-sn}.}
    \label{fig:zimmermann}
\end{figure}

Zimmermann \etal provided results on their own datasets, defining the skeleton as a set of 18 joints, which are based on the definition of the COCO dataset keypoints~\cite{lin2014microsoft}. In order to compare our results with those yielded by Zimmermann's approach, we have averaged their \textit{LEar} and \textit{REar} keypoints and \textit{RHip} and \textit{LHip} keypoints to get unique head and pelvis labels, respectively. By doing so, we match the modifications presented in Section~\ref{sec:benchmark_dataset} for \gls{bmhad} and MPII datasets.

For some frames in the evaluated scenes, specially in scenarios with high occlusions, Zimmermann's method did not return any estimated location for certain joints. When computing the \gls{pckh} score, as it is simply a percentage of parts correctly detected, these joints can be included in the evaluation and are taken into account as \textit{not correct}. However, as \gls{mpjpe} is defined as an average of distances, such empty estimations are invalid and must be discarded. In order to avoid a biased comparison, such joint estimations have been removed for every method when measuring the \gls{mpjpe}. For this comparison, we have used the \gls{bmhad} dataset in its entirety.

Taking a look at the \gls{pckh} curves shown in Figure~\ref{fig:pckh_3d}, we conclude that, except for the pelvis joint, the overall performance of Zimmermann's method is similar to ours, specially when using \glspl{cpm} as our 2D pose estimator. In fact, Zimmermann's method only performs better for shoulders and pelvis joints, with a total \gls{pckh}@1 of 81.55\%, versus an 80.23\% of our method with \glspl{cpm} (see Table~\ref{tab:pckh_3d_joint}). It is also important to note that, when evaluating 3D estimations with \gls{pckh}, final scores are lower than when evaluating 2D locations, as the added dimension increases the magnitude of the error, even though the head segment length also increases.

\begin{figure}[h]
    \centering
    \includegraphics[width=\textwidth]{figures/pckh_3d.png}
    \caption{Quantitative 3D pose estimation results on BMHAD per joint, using as figure of merit PCKh~3D~@~0-1~(\%).}
    \label{fig:pckh_3d}
\end{figure}

\begin{table}[!ht]  
  \centering
  \resizebox{\textwidth}{!}{\begin{tabular}{l||ccccccc|c}
    \hline
    & Head & Shoulders & Elbows & Hands & Pelvis & Knees & Feet & TOTAL \\
    \hline
    \glspl{cpm} & \textbf{91.93} & 86.05 & \textbf{75.59} & \textbf{61.95} & 68.17 & \textbf{89.93} & \textbf{87.84} & 80.23 \\
    \gls{sh} & 90.71 & 85.36 & 74.54 & 60.76 & 67.85 & 87.20 & 85.06 & 78.70 \\
    \glspl{cp} & 87.41 & 83.32 & 70.33 & 54.08 & 66.87 & 84.49 & 79.94 & 74.88 \\
    Zimm. & 89.13 & \textbf{88.05} & 74.61 & 61.50 & \textbf{91.71} & 87.50 & 87.21 & \textbf{81.55} \\
    \hline
  \end{tabular}}
  \caption{Quantitative 3D pose estimation results on BMHAD per joint, using as figure of merit PCKh~3D~@~1~(\%). Values in bold correspond to the best results achieved for each category.}
  \label{tab:pckh_3d_joint}
\end{table}

If we take a look at the results obtained for the same estimations when evaluated in terms of \gls{mpjpe}, we see that the gap between our method and Zimmerman's increases. First of all, boxplots in Figure~\ref{fig:mpjpe} show that Zimmermann's method error present a lower standard deviation for most of the joints. Moreover, according to Table~\ref{tab:mpjpe_joint}, Zimmermann's mean error is lower for every joint, with the exceptions of head and feet, with a total \gls{mpjpe} of 128.1 mm, versus a \gls{mpjpe} of 144.1 mm for our method with \glspl{cpm}. 

\begin{figure}[h]
    \centering
    \includegraphics[width=\textwidth]{figures/mpjpe.png}
    \caption{Quantitative 3D pose estimation results on BMHAD per joint, using as figure of merit the \emph{Euclidean} distance between estimations and groun-truth (mm).}
    \label{fig:mpjpe}
\end{figure}

\begin{table}[!ht]  
  \centering
  \resizebox{\textwidth}{!}{\begin{tabular}{l||ccccccc|c}
    \hline
    & Head & Shoulders & Elbows & Hands & Pelvis & Knees & Feet & TOTAL \\
    \hline
    \glspl{cpm} & \textbf{99.5} & 129.4 & 149.9 & 200.5 & 179.9 & 132.8 & \textbf{112.5} & 144.1 \\
    \gls{sh} & 102.3 & 124.8 & 152.0 & 208.7 & 179.0 & 138.6 & 127.2 & 148.7 \\
    \glspl{cp} & 153.6 & 127.2 & 167.9 & 232.2 & 178.9 & 148.5 & 147.1 & 164.8 \\
    Zimm. & 101.2 & \textbf{102.1} & \textbf{143.5} & \textbf{198.7} & \textbf{95.1} & \textbf{108.2} & 117.7 & \textbf{128.1} \\
    \hline
  \end{tabular}}
  \caption{Quantitative 3D pose estimation results on BMHAD per joint, using as figure of merit MPJPE (mm). Values in bold correspond to the best results achieved for each category.}
  \label{tab:mpjpe_joint}
\end{table}

It is still a very competitive result for our approach, particularly taking into account its simplicity and low computational burden, which will be explored in Section~\ref{sec:computational_burden}. Also, our explicit outlier rejection and Kalman filtering plays a vital role in the final results, as evidenced by the differences shown in the \gls{pckh} and \gls{mpjpe} scores. Even when our average error in terms of distance was higher than the one yielded by Zimmermann's method, we manage to get very similar or even better percentages of correct parts, which means that joints estimated with broad error were properly identified and corrected. In Figure~\ref{fig:zimmermann_failed}, some examples of such cases are presented.

\begin{figure}[h]\centering
    \begin{subfigure}{0.49\textwidth}\centering
        \includegraphics[width=\textwidth]{figures/zimmermann_failed.png} 
        \caption{Outliers found in Zimmermann's estimations.}
        \label{subfig:zimmermann_failed}
    \end{subfigure}
    \begin{subfigure}{0.49\textwidth}\centering
        \includegraphics[width=\textwidth]{figures/ours_right.png}
        \caption{Our results for the same frames, using CPMs as 2D estimator.}
        \label{subfig:ours_cpm_right}
    \end{subfigure}
    \caption{Our outlier rejection and filtering mechanisms help alleviate strong deviations like the ones shown in (a). By doing so, even if the overall error in favorable conditions is higher, we get plausible poses consistently (b).}
    \label{fig:zimmermann_failed}
\end{figure}

Regarding the wide difference in terms of pelvis estimation, it is justified by the nature of our method. While Zimmermann's method is trained with labeled 3D joint locations, ours rely on no more than a stream of depth images, from a single point of view, to get the 3D locations from the 2D estimations. With that approach, we cannot \textit{see through} the human body and the only information that we have is the closest point to the camera in the scene when we trace a ray between the sensor and the 2D estimation. Figure~\ref{fig:thickness} exemplifies how this effect might have an impact on our estimations. In that regard, the joint for which this assumption is less precise is the pelvis, as it is placed in a region of the human body thicker than the other keypoints considered, and depending on how the subject is placed on the scene, this thickness changes. In general, our algorithm is more prone to failure if the subject is self-occluded, as it can be see in Figure~\ref{fig:selfocclusion}. A possible solution that might be explored in the future in order to correct this kind of deviations is simply including in our estimations the average thickness of each human body part detected.

\begin{figure}[h]
    \centering
    \includegraphics[width=\textwidth]{figures/thickness.png}
    \caption{As our 3D estimations rely on depth images captured from a single point of view and no prior about the human body is imposed, we might commit a constant error when detecting joints that are not labeled close to the body surface.}
    \label{fig:thickness}
\end{figure}

\begin{figure}[h]\centering
    \begin{subfigure}{0.49\textwidth}\centering
        \includegraphics[width=\textwidth]{figures/zimmermann_selfocclusion.png} 
        \caption{Zimmermann's estimation.}
        \label{subfig:zimmermann_selfocclusion}
    \end{subfigure}
    \begin{subfigure}{0.49\textwidth}\centering
        \includegraphics[width=\textwidth]{figures/ours_selfocclusion.png}
        \caption{Our estimation (using CPMs).}
        \label{subfig:ours_selfocclusion}
    \end{subfigure}
    \caption{While Zimmerman's method (a) is able to deal with self-occlusions in the scene, our algorithm (b) relies only on the depth information without any learned prior on the human pose configuration, which causes estimation errors under these circumstances.}
    \label{fig:selfocclusion}
\end{figure}

The same imbalance in \gls{pckh} and \gls{mpjpe} results mentioned in the per joint analysis can be seen when evaluating per action. While our method performs better in three out of eleven actions in terms of \gls{pckh}, it only outperforms Zimmermann's in one of the scenes when evaluating \gls{mpjpe} (see Table~\ref{tab:mpjpe_action} and Table~\ref{tab:pckh_3d_action}).

\begin{table}[!ht]  
  \centering
  \resizebox{\textwidth}{!}{\begin{tabular}{l||oorobbbrggg|c}
    \hline
    & a01 & a02 & a03 & a04 & a05 & a06 & a07 & a08 & a09 & a10 & a11 & TOTAL \\
    \hline
    \glspl{cpm} & 124.3 & 123.8 & \textbf{159.7} & 156.1 & 117.5 & 116.9 & 139.2 & 135.8 & 173.6 & 178.0 & 166.1 & 144.1 \\
    \gls{sh} & 123.7 & 119.8 & 163.1 & 158.1 & 122.6 & 128.9 & 141.8 & 146.4 & 179.7 & 186.5 & 171.6 & 148.7 \\
    \glspl{cp} & 138.6 & 150.5 & 181.0 & 172.5 & 146.6 & 146.4 & 155.2 & 171.0 & 185.9 & 206.0 & 175.7 & 164.8 \\
    Zimm. & \textbf{105.8} & \textbf{114.5} & 179.4 & \textbf{120.6} & \textbf{115.9} & \textbf{102.7} & \textbf{116.0} & \textbf{118.0} & \textbf{137.4} & \textbf{157.6} & \textbf{133.7} & \textbf{128.1} \\
    \hline
  \end{tabular}}
  \caption{Quantitative 3D pose estimation results on BMHAD per action, using as figure of merit MPJPE (mm). Values in bold correspond to the best results achieved for each category. Columns are colored following the convention established in Section~\ref{sec:benchmark_dataset}.}
  \label{tab:mpjpe_action}
\end{table}

\begin{table}[!ht]  
  \centering
  \resizebox{\textwidth}{!}{\begin{tabular}{l||oorobbbrggg|c}
    \hline
    & a01 & a02 & a03 & a04 & a05 & a06 & a07 & a08 & a09 & a10 & a11 & TOTAL \\
    \hline
    \glspl{cpm} & 89.48 & 91.16 & \textbf{71.25} & 78.15 & \textbf{92.94} & 91.57 & 84.67 & 84.73 & 70.54 & 67.69 & 73.03 & 80.23 \\
    \gls{sh} & 89.39 & \textbf{91.99} & 65.69 & 77.54 & 92.12 & 90.08 & 83.33 & 82.10 & 70.36 & 67.32 & 72.42 & 78.70 \\
    \glspl{cp} & 86.29 & 85.18 & 60.46 & 75.21 & 85.60 & 84.45 & 81.59 & 79.21 & 69.00 & 64.52 & 71.12 & 74.88 \\
    Zimm. & \textbf{94.15} & 90.52 & 55.91 & \textbf{87.83} & 89.25 & \textbf{93.07} & \textbf{89.06} & \textbf{88.57} & \textbf{81.97} & \textbf{73.02} & \textbf{80.78} & \textbf{81.55} \\
    \hline
  \end{tabular}}
  \caption{Quantitative 3D pose estimation results on BMHAD per action, using as figure of merit PCKh~3D~@~1~(\%). Values in bold correspond to the best results achieved for each category. Columns are colored following the convention established in Section~\ref{sec:benchmark_dataset}.}
  \label{tab:pckh_3d_action}
\end{table}

If we compare results after grouping the actions in terms of dynamics and occlusion, it is shown that our method performs better in the most challenging scenarios, \ie very dynamic actions with occluded joints. This observation holds true for both \gls{pckh} in Table~\ref{tab:pckh_3d_action_type} and \gls{mpjpe} in Table~\ref{tab:mpjpe_action_type}. These differences can be caused again by our outlier rejection and Kalman filtering stages, which reduce the number of greatly deviated estimations in an explicit manner. Meanwhile, Zimmermann's method does not apply any post-processing on the yielded estimations and so great errors might appear when, for instance, a joint gets out of view or the subject is moving too fast. 

\begin{table}[!ht]  
  \centering
  \resizebox{\textwidth}{!}{\begin{tabular}{l||ro|gb}
    \hline
    & \multicolumn{2}{c|}{High Dynamics} & \multicolumn{2}{c}{Low Dynamics} \\
    \cline{2-5}
    & Heavily Occluded & Slightly Occluded & Heavily Occluded & Slightly Occluded \\
    \hline
    \glspl{cpm} & \textbf{77.99} & 86.26 & 70.42 & 89.73 \\
    \gls{sh} & 73.90 & 86.31 & 70.03 & 88.51 \\
    \glspl{cp} & 69.83 & 82.22 & 68.21 & 83.88 \\
    Zimm. & 72.24 & \textbf{90.83} & \textbf{78.59} & \textbf{90.46} \\
    \hline
  \end{tabular}}
  \caption{Quantitative 3D pose estimation results on BMHAD per action type, using as figure of merit PCKh~3D~@~1~(\%). Values in bold correspond to the best results achieved for each category. Columns are colored following the convention established in Section~\ref{sec:benchmark_dataset}.}
  \label{tab:pckh_3d_action_type}
\end{table}

\begin{table}[!ht]  
  \centering
  \resizebox{\textwidth}{!}{\begin{tabular}{l||ro|gb}
    \hline
    & \multicolumn{2}{c|}{High Dynamics} & \multicolumn{2}{c}{Low Dynamics} \\
    \cline{2-5}
    & Heavily Occluded & Slightly Occluded & Heavily Occluded & Slightly Occluded \\
    \hline
    \glspl{cpm} & \textbf{147.8} & 134.7 & 172.6 & 124.5 \\
    \gls{sh} & 154.8 & 133.9 & 179.3 & 131.1 \\
    \glspl{cp} & 176.0 & 153.9 & 189.2 & 149.4 \\
    Zimm. & 148.7 & \textbf{113.6} & \textbf{142.9} & \textbf{111.6} \\
    \hline
  \end{tabular}}
  \caption{Quantitative 3D pose estimation results on BMHAD per action type, using as figure of merit MPJPE (mm). Values in bold correspond to the best results achieved for each category. Columns are colored following the convention established in Section~\ref{sec:benchmark_dataset}.}
  \label{tab:mpjpe_action_type}
\end{table}

An interesting lesson from the results we have just discussed is that our method for filtering 3D estimations could easily be plugged in after another 3D human pose estimation method such as Zimmermann's, further improving their results. By doing so, we would increase the computational cost, which is a major requirement in this work, but whether this could be a reasonable solution or not depends on the application.

\section{Computational burden}\label{sec:computational_burden}
Besides achieving very competitive quantitative results when compared with the \gls{sota} method introduced by Zimmermann \etal\cite{Zimmermann2018-sn}, we keep the computational costs as low as possible by using very lightweight algorithms for the 2D to 3D augmentation of the estimated coordinates. In Table~\ref{tab:computational_burden}, a comparison of the computational burden associated with each evaluated method in previous sections is presented. All the tests have been performed after evaluating 2108 frames of 10 different scenes extracted from \gls{bmhad}, using a PC with the following specifications:

\begin{itemize}
    \item \textbf{RAM memory}: 12Gb.
    \item \textbf{GPU}: Nvidia GeForce GTX 1050.
    \item \textbf{CPU}: Intel Core i7-7700HQ @ 2.80GHz
\end{itemize}

\begin{table}[!ht]  
  \centering
  \resizebox{\textwidth}{!}{\begin{tabular}{l||cccc|c}
    \hline
    & \glspl{cpm} & \gls{sh} & \glspl{cp} & Zimm. & Our 2D to 3D registration module alone\\
    \hline
    \gls{fps} & 8.11 & \textbf{16.07} & 10.64 & 2.22 & \textbf{99.76}\\
    \hline
  \end{tabular}}
  \caption{Computational cost of each method, measured as the average number of frames per second.}
  \label{tab:computational_burden}
\end{table}

As it can be seen, our methods perform with much higher framerates, which facilitates their inclusion in robotic systems as embedded software. For instance, our method is around four times faster than Zimmermann's when using \glspl{cpm} as 2D estimator, and more than seven times faster when using \gls{sh}. When measuring the time that it takes to carry out the 2D to 3D step, without taking into account the 2D estimation process, our algorithm only adds 0.01s to the total computation time for a single frame.

It is also worth mentioning that the computational burden can be further reduced depending on the boxsize of the input image. Depending on the accuracy required by the application, an optimal trade-off between the quality of the estimation and the computational cost may be explored.

\section{Qualitative evaluation: real-time demo}\label{sec:demo}
We have already discussed how our method compares with the \gls{sota}, both in terms of accuracy and computational burden. Our approach cannot outperform the most recent works in the literature in 3D human pose estimation, but it is competitive and much faster. However, the tests carried out have not assessed its performance with more common scenes \emph{in the wild}. For that purpose, we have developed a real-time demo that works in an \emph{off-the-shelf} computer with a commercial RGBD sensor. The set-up used and the system designed for that demo are presented in this section, along with a discussion about the obtained results.

In Figure~\ref{fig:setup}, the simple set-up built for this demo is presented. Our demo has been run on a PC with the specifications described in Section~\ref{tab:computational_burden}. This PC is equipped with an Nvidia GPU~\cite{nvidia}, which allows the 2D estimator to make use of the CUDA parallel computing platform~\cite{nickolls2008scalable} for accelerating the inference of the human poses. The RGBD sensor used for the demo is an Asus Xtion PRO LIVE~\cite{xtion}. This camera captures RGBD video sequences at 30\gls{fps} with a resolution of 640x480 pixels and a working distance that ranges between 0.5 and 3.5 meters. It relies on an infrared sensor and structured light projected on the scene for estimating depth.

\begin{figure}[h]\centering
    \begin{subfigure}{\textwidth}\centering
        \includegraphics[width=\textwidth]{figures/real_setup.png} 
        \caption{Image from the real set-up.}
        \label{subfig:real_setup}
    \end{subfigure}
    \begin{subfigure}{\textwidth}\centering
        \includegraphics[width=\textwidth]{figures/setup.png}
        \caption{Simplified diagram with relevant distances.}
        \label{subfig:simple_setup}
    \end{subfigure}
    \caption{Diagram of the set-up used for the real-time demo.}
    \label{fig:setup}
\end{figure}

Regarding the software used for the demo, our system employs ROS~\cite{ros} for PC-camera communication. In particular, we have used the \emph{openni2} node~\cite{openni}, which allows depth and RGBD image registration and distortion correction, based on our camera calibration data. Our demo application has been entirely developed in \emph{Python} in a threaded manner. A thread is in charge of receiving and serving the color and depth images provided by the ROS node. Another thread updates a GUI that shows in real time these images and the result of the 2D pose estimation. Last but not least, a third thread is in charge of estimating the 2D poses and, from these, generating the final 3D estimations as described in Section~\ref{sec:3d_estimation}. These 3D estimations can then be inspected in real-time thanks to the 3D visualizer gently shared by Roberto Pérez~\cite{perez_gonzalez_2019}. This visualizer has been built as a desktop app developed with Electron~\cite{electron}. It is important to note that, for this demo, we have used \glspl{cpm} as 2D estimator. In order to further improve the framerate we have used a boxsize of 256x256, which yields good enough results. A complete diagram of the developed software can be seen in Figure~\ref{fig:sw}. The  developed application is publicly available and hosted in \emph{GitHub}~\footnote{\url{https://github.com/RoboticsLabURJC/2017-tfm-david-pascual/}}.% the figure must include screenshots from the GUIs

\begin{figure}[h]
    \centering
    \includegraphics[width=\textwidth]{figures/sw.png}
    \caption{Overview of the application developed.}
    \label{fig:sw}
\end{figure}

Results for this demo are in Figure~\ref{fig:demo_result}. The application runs fluently in real-time and, in general, achieves good results for the tested poses. It is worth mentioning that we identify failures in certain situations, \eg when the subject is too close to the camera or stands sidewards, which increases the number of occluded joints. Such cases can be seen in Figure~\ref{fig:demo_failures}. A complete video showcasing the described demo is publicly available on \emph{YouTube}~\footnote{\url{https://www.youtube.com/watch?v=W3XirsadmNg}}.

\begin{figure}[h]\centering
    \begin{subfigure}{\textwidth}\centering
        \includegraphics[width=\textwidth]{figures/gui_3d.png} 
        \caption{3D estimation results.}
        \label{subfig:gui_3d}
    \end{subfigure}
    \begin{subfigure}{\textwidth}\centering
        \includegraphics[width=\textwidth]{figures/gui_2d.png}
        \caption{2D estimation results.}
        \label{subfig:gui_2d}
    \end{subfigure}
    \caption{Real-time results as presented by our application for (a) 3D pose estimation and (b) 2D estimation.}
    \label{fig:demo_result}
\end{figure}

\begin{figure}[h]\centering
    \begin{subfigure}{0.49\textwidth}\centering
        \includegraphics[width=0.99\textwidth]{figures/too_close.png} 
        \caption{Estimation fails due to subject proximity.}
        \label{subfig:too_close}
    \end{subfigure}
    \begin{subfigure}{0.49\textwidth}\centering
        \includegraphics[width=0.99\textwidth]{figures/occlusion.png}
        \caption{Estimation in auto-occlusion.}
        \label{subfig:occlusion}
    \end{subfigure}
    \caption{Our algorithm might fail in challenging scenarios, such as (a) the subject being too close to the camera or (b) self-occlusions.}
    \label{fig:demo_failures}
\end{figure}