\chapter*{Resumen}
En los últimos años se ha producido un proceso de digitalización y aumento de la capacidad de cómputo sin precedentes. Todos los ámbitos de la sociedad se han visto afectados por la llamada \emph{4ª Revolución Industrial}, especialmente aquellas áreas de conocimiento más transversales como la robótica, cuyas aplicaciones son embebidas habitualmente en dispositivos con recursos limitados y deben funcionar en tiempo real.

Uno de los campos de estudio más relevantes dentro de la róbotica es la \emph{Interacción Humano-Robot}, presente siempre que el robot requiera tener en cuenta a las personas de su entorno para planificar sus acciones. En consonancia con nuestras formas de comunicación, esta tarea suele depender del procesamiento de señales de audio o vídeo, aunque esta última tiene la ventaja de no requerir interacción explícita por parte de dichas personas.

Nuestro trabajo se centra en la estimación de pose en humanos, un área de la visión artificial estrechamente relacionada con la \emph{Interacción Humano-Robot}. Históricamente resuelta mediante complejos sistemas de captura de movimiento, en las últimas décadas se ha dedicado un gran esfuerzo al desarrollo de algoritmos sobre dispositivos más escalables. La inclusión de técnicas de aprendizaje profundo ha catapultado la efectividad de estos últimos. Sin embargo, la cantidad de ejemplos etiquetados y recursos computacionales requeridos puede ser prohibitiva, en especial cuando se trabaja con datos tridimensionales, que proporcionan al robot una información más completa sobre su entorno.

Teniendo estos retos en cuenta, presentamos un sistema completo para la estimación 3D de pose en humanos. El método propuesto funciona en tiempo real haciendo uso de un ordenador y un sensor RGBD comerciales. En este sentido, nuestras principales contribuciones son la evaluación cuantitativa de diferentes estimadores de pose 2D, el diseño y desarrollo del sistema propuesto, la evaluación cuantitativa de dicho sistema, incluyendo comparativas con el estado del arte, y el desarrollo de un demostrador en tiempo real que valida expirementalmente su rendimiento a nivel cualitativo.

Los resultados muestran que nuestra solución es competitiva en términos de precisión y suficientemente ligera para funcionar en tiempo real, como queda demostrado a través de la evaluación cuantitativa y cualitativa. Concluimos, por tanto, que su uso es apropiado en aplicaciones robóticas que requieren interacción con humanos.